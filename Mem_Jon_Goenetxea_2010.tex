\documentclass[12pt,letterpaper]{article}
\usepackage[spanish]{babel}
\usepackage[ansinew]{inputenc}
\usepackage[right=2cm,left=3cm,top=2cm,bottom=2cm,headsep=0cm,footskip=0.5cm]{geometry}
\usepackage[dvips]{graphicx}

%Introducci�n
%Estado del arte
%Bases teoricas necesarias
%Experimentos realizados/Fases del trabajo
%Trabajo terminado
%Conclusiones
%L�neas futuras

\begin{document}

% first the title is needed
\title{Ikusmen artifizialerako interakzioa hiru dimentsiotako analisiari zuzendua}

% a short form should be given in case it is too long for the running head
%\titlerunning{Hiru dimentsiotako analisi eta interakzioa}

\author{Jon Goenetxea}
%\thanks{Eskerrak}%


\section{Sarrera}
%\abstract{Sarrera}
Lan honetan, konputagailuen eta gizakiaren arteko interakziorako teknologiak azterturko dira. Horien artea, hiru dimentsiotako errepresentazioaren adibide bat garatuko da, non errepresentazioaren irudia, pantailaren aurrean mantenduko den, ikuskatzailearen mugiemenduak kontuan izanik irudiaren perspectiba aldatzeko. 

Bestalde, martxan dagoen aplikazioari aginduak hemateko, eta honekin interakzioa izateko, gestikulazioan oinarritzen den ``interface'' baten garapena egin da. Gestu hauen bitartez, erabiltzaileak aplikazioaren parametroak aldatu ditzake, errepresentazioa mugitu, aldatu edo ta transformatzeko.

Errepresentazioa egiteko, bi atze-proiektoreko mantaila bat erabili da, bi irudi polarizatu proiektatuz. Erabiltzaileak betaurreko polarizatuak erabili behar baditu ere, merkatuan dagoen teknologia honek emaitza onak hematen ditu. Interakziorako, Nintendo-ren WiiRemote Controler urruti controla erabili da. Bertan integraturik dituen azelerazio neurgailuei esker, gestuen datuak lortzea posible da, eta aurrepartean duen kamera infragorria erabilita, puntero bezala erabili daiteke.

\section{Estado del arte***}
\subsection{3 Dimentsiotako grafikoak}

Mundu guztiak ikusi du noizbait ordenagailu batekin eginiko hiru dimentsiotako errepresentazioren bat. Bideojoko, zinema, industria eta beste zenbait alorretan oso erabiliak diren herraminta hauek, ia egitazkoak diruditen irudiak sortzeko aukera hematen dio adituari.
Teknologia honen garapena 60. hamarkadan hasi zen Iv�n Sutherland-ek bere lana argitaratu zuenean MIT institutuan. Lan honetan, CRT monitore baten marra bat marrazten zuen programa baten oinarritzen zen.
Hurrengo hamarkadan, 


\end{document}